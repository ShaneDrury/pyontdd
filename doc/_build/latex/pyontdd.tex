% Generated by Sphinx.
\def\sphinxdocclass{report}
\documentclass[letterpaper,10pt,english]{sphinxmanual}
\usepackage[utf8]{inputenc}
\DeclareUnicodeCharacter{00A0}{\nobreakspace}
\usepackage{cmap}
\usepackage[T1]{fontenc}
\usepackage{babel}
\usepackage{times}
\usepackage[Bjarne]{fncychap}
\usepackage{longtable}
\usepackage{sphinx}
\usepackage{multirow}


\title{pyontdd Documentation}
\date{August 08, 2013}
\release{0.1.1}
\author{Shane Drury}
\newcommand{\sphinxlogo}{}
\renewcommand{\releasename}{Release}
\makeindex

\makeatletter
\def\PYG@reset{\let\PYG@it=\relax \let\PYG@bf=\relax%
    \let\PYG@ul=\relax \let\PYG@tc=\relax%
    \let\PYG@bc=\relax \let\PYG@ff=\relax}
\def\PYG@tok#1{\csname PYG@tok@#1\endcsname}
\def\PYG@toks#1+{\ifx\relax#1\empty\else%
    \PYG@tok{#1}\expandafter\PYG@toks\fi}
\def\PYG@do#1{\PYG@bc{\PYG@tc{\PYG@ul{%
    \PYG@it{\PYG@bf{\PYG@ff{#1}}}}}}}
\def\PYG#1#2{\PYG@reset\PYG@toks#1+\relax+\PYG@do{#2}}

\expandafter\def\csname PYG@tok@gd\endcsname{\def\PYG@tc##1{\textcolor[rgb]{0.63,0.00,0.00}{##1}}}
\expandafter\def\csname PYG@tok@gu\endcsname{\let\PYG@bf=\textbf\def\PYG@tc##1{\textcolor[rgb]{0.50,0.00,0.50}{##1}}}
\expandafter\def\csname PYG@tok@gt\endcsname{\def\PYG@tc##1{\textcolor[rgb]{0.00,0.27,0.87}{##1}}}
\expandafter\def\csname PYG@tok@gs\endcsname{\let\PYG@bf=\textbf}
\expandafter\def\csname PYG@tok@gr\endcsname{\def\PYG@tc##1{\textcolor[rgb]{1.00,0.00,0.00}{##1}}}
\expandafter\def\csname PYG@tok@cm\endcsname{\let\PYG@it=\textit\def\PYG@tc##1{\textcolor[rgb]{0.25,0.50,0.56}{##1}}}
\expandafter\def\csname PYG@tok@vg\endcsname{\def\PYG@tc##1{\textcolor[rgb]{0.73,0.38,0.84}{##1}}}
\expandafter\def\csname PYG@tok@m\endcsname{\def\PYG@tc##1{\textcolor[rgb]{0.13,0.50,0.31}{##1}}}
\expandafter\def\csname PYG@tok@mh\endcsname{\def\PYG@tc##1{\textcolor[rgb]{0.13,0.50,0.31}{##1}}}
\expandafter\def\csname PYG@tok@cs\endcsname{\def\PYG@tc##1{\textcolor[rgb]{0.25,0.50,0.56}{##1}}\def\PYG@bc##1{\setlength{\fboxsep}{0pt}\colorbox[rgb]{1.00,0.94,0.94}{\strut ##1}}}
\expandafter\def\csname PYG@tok@ge\endcsname{\let\PYG@it=\textit}
\expandafter\def\csname PYG@tok@vc\endcsname{\def\PYG@tc##1{\textcolor[rgb]{0.73,0.38,0.84}{##1}}}
\expandafter\def\csname PYG@tok@il\endcsname{\def\PYG@tc##1{\textcolor[rgb]{0.13,0.50,0.31}{##1}}}
\expandafter\def\csname PYG@tok@go\endcsname{\def\PYG@tc##1{\textcolor[rgb]{0.20,0.20,0.20}{##1}}}
\expandafter\def\csname PYG@tok@cp\endcsname{\def\PYG@tc##1{\textcolor[rgb]{0.00,0.44,0.13}{##1}}}
\expandafter\def\csname PYG@tok@gi\endcsname{\def\PYG@tc##1{\textcolor[rgb]{0.00,0.63,0.00}{##1}}}
\expandafter\def\csname PYG@tok@gh\endcsname{\let\PYG@bf=\textbf\def\PYG@tc##1{\textcolor[rgb]{0.00,0.00,0.50}{##1}}}
\expandafter\def\csname PYG@tok@ni\endcsname{\let\PYG@bf=\textbf\def\PYG@tc##1{\textcolor[rgb]{0.84,0.33,0.22}{##1}}}
\expandafter\def\csname PYG@tok@nl\endcsname{\let\PYG@bf=\textbf\def\PYG@tc##1{\textcolor[rgb]{0.00,0.13,0.44}{##1}}}
\expandafter\def\csname PYG@tok@nn\endcsname{\let\PYG@bf=\textbf\def\PYG@tc##1{\textcolor[rgb]{0.05,0.52,0.71}{##1}}}
\expandafter\def\csname PYG@tok@no\endcsname{\def\PYG@tc##1{\textcolor[rgb]{0.38,0.68,0.84}{##1}}}
\expandafter\def\csname PYG@tok@na\endcsname{\def\PYG@tc##1{\textcolor[rgb]{0.25,0.44,0.63}{##1}}}
\expandafter\def\csname PYG@tok@nb\endcsname{\def\PYG@tc##1{\textcolor[rgb]{0.00,0.44,0.13}{##1}}}
\expandafter\def\csname PYG@tok@nc\endcsname{\let\PYG@bf=\textbf\def\PYG@tc##1{\textcolor[rgb]{0.05,0.52,0.71}{##1}}}
\expandafter\def\csname PYG@tok@nd\endcsname{\let\PYG@bf=\textbf\def\PYG@tc##1{\textcolor[rgb]{0.33,0.33,0.33}{##1}}}
\expandafter\def\csname PYG@tok@ne\endcsname{\def\PYG@tc##1{\textcolor[rgb]{0.00,0.44,0.13}{##1}}}
\expandafter\def\csname PYG@tok@nf\endcsname{\def\PYG@tc##1{\textcolor[rgb]{0.02,0.16,0.49}{##1}}}
\expandafter\def\csname PYG@tok@si\endcsname{\let\PYG@it=\textit\def\PYG@tc##1{\textcolor[rgb]{0.44,0.63,0.82}{##1}}}
\expandafter\def\csname PYG@tok@s2\endcsname{\def\PYG@tc##1{\textcolor[rgb]{0.25,0.44,0.63}{##1}}}
\expandafter\def\csname PYG@tok@vi\endcsname{\def\PYG@tc##1{\textcolor[rgb]{0.73,0.38,0.84}{##1}}}
\expandafter\def\csname PYG@tok@nt\endcsname{\let\PYG@bf=\textbf\def\PYG@tc##1{\textcolor[rgb]{0.02,0.16,0.45}{##1}}}
\expandafter\def\csname PYG@tok@nv\endcsname{\def\PYG@tc##1{\textcolor[rgb]{0.73,0.38,0.84}{##1}}}
\expandafter\def\csname PYG@tok@s1\endcsname{\def\PYG@tc##1{\textcolor[rgb]{0.25,0.44,0.63}{##1}}}
\expandafter\def\csname PYG@tok@gp\endcsname{\let\PYG@bf=\textbf\def\PYG@tc##1{\textcolor[rgb]{0.78,0.36,0.04}{##1}}}
\expandafter\def\csname PYG@tok@sh\endcsname{\def\PYG@tc##1{\textcolor[rgb]{0.25,0.44,0.63}{##1}}}
\expandafter\def\csname PYG@tok@ow\endcsname{\let\PYG@bf=\textbf\def\PYG@tc##1{\textcolor[rgb]{0.00,0.44,0.13}{##1}}}
\expandafter\def\csname PYG@tok@sx\endcsname{\def\PYG@tc##1{\textcolor[rgb]{0.78,0.36,0.04}{##1}}}
\expandafter\def\csname PYG@tok@bp\endcsname{\def\PYG@tc##1{\textcolor[rgb]{0.00,0.44,0.13}{##1}}}
\expandafter\def\csname PYG@tok@c1\endcsname{\let\PYG@it=\textit\def\PYG@tc##1{\textcolor[rgb]{0.25,0.50,0.56}{##1}}}
\expandafter\def\csname PYG@tok@kc\endcsname{\let\PYG@bf=\textbf\def\PYG@tc##1{\textcolor[rgb]{0.00,0.44,0.13}{##1}}}
\expandafter\def\csname PYG@tok@c\endcsname{\let\PYG@it=\textit\def\PYG@tc##1{\textcolor[rgb]{0.25,0.50,0.56}{##1}}}
\expandafter\def\csname PYG@tok@mf\endcsname{\def\PYG@tc##1{\textcolor[rgb]{0.13,0.50,0.31}{##1}}}
\expandafter\def\csname PYG@tok@err\endcsname{\def\PYG@bc##1{\setlength{\fboxsep}{0pt}\fcolorbox[rgb]{1.00,0.00,0.00}{1,1,1}{\strut ##1}}}
\expandafter\def\csname PYG@tok@kd\endcsname{\let\PYG@bf=\textbf\def\PYG@tc##1{\textcolor[rgb]{0.00,0.44,0.13}{##1}}}
\expandafter\def\csname PYG@tok@ss\endcsname{\def\PYG@tc##1{\textcolor[rgb]{0.32,0.47,0.09}{##1}}}
\expandafter\def\csname PYG@tok@sr\endcsname{\def\PYG@tc##1{\textcolor[rgb]{0.14,0.33,0.53}{##1}}}
\expandafter\def\csname PYG@tok@mo\endcsname{\def\PYG@tc##1{\textcolor[rgb]{0.13,0.50,0.31}{##1}}}
\expandafter\def\csname PYG@tok@mi\endcsname{\def\PYG@tc##1{\textcolor[rgb]{0.13,0.50,0.31}{##1}}}
\expandafter\def\csname PYG@tok@kn\endcsname{\let\PYG@bf=\textbf\def\PYG@tc##1{\textcolor[rgb]{0.00,0.44,0.13}{##1}}}
\expandafter\def\csname PYG@tok@o\endcsname{\def\PYG@tc##1{\textcolor[rgb]{0.40,0.40,0.40}{##1}}}
\expandafter\def\csname PYG@tok@kr\endcsname{\let\PYG@bf=\textbf\def\PYG@tc##1{\textcolor[rgb]{0.00,0.44,0.13}{##1}}}
\expandafter\def\csname PYG@tok@s\endcsname{\def\PYG@tc##1{\textcolor[rgb]{0.25,0.44,0.63}{##1}}}
\expandafter\def\csname PYG@tok@kp\endcsname{\def\PYG@tc##1{\textcolor[rgb]{0.00,0.44,0.13}{##1}}}
\expandafter\def\csname PYG@tok@w\endcsname{\def\PYG@tc##1{\textcolor[rgb]{0.73,0.73,0.73}{##1}}}
\expandafter\def\csname PYG@tok@kt\endcsname{\def\PYG@tc##1{\textcolor[rgb]{0.56,0.13,0.00}{##1}}}
\expandafter\def\csname PYG@tok@sc\endcsname{\def\PYG@tc##1{\textcolor[rgb]{0.25,0.44,0.63}{##1}}}
\expandafter\def\csname PYG@tok@sb\endcsname{\def\PYG@tc##1{\textcolor[rgb]{0.25,0.44,0.63}{##1}}}
\expandafter\def\csname PYG@tok@k\endcsname{\let\PYG@bf=\textbf\def\PYG@tc##1{\textcolor[rgb]{0.00,0.44,0.13}{##1}}}
\expandafter\def\csname PYG@tok@se\endcsname{\let\PYG@bf=\textbf\def\PYG@tc##1{\textcolor[rgb]{0.25,0.44,0.63}{##1}}}
\expandafter\def\csname PYG@tok@sd\endcsname{\let\PYG@it=\textit\def\PYG@tc##1{\textcolor[rgb]{0.25,0.44,0.63}{##1}}}

\def\PYGZbs{\char`\\}
\def\PYGZus{\char`\_}
\def\PYGZob{\char`\{}
\def\PYGZcb{\char`\}}
\def\PYGZca{\char`\^}
\def\PYGZam{\char`\&}
\def\PYGZlt{\char`\<}
\def\PYGZgt{\char`\>}
\def\PYGZsh{\char`\#}
\def\PYGZpc{\char`\%}
\def\PYGZdl{\char`\$}
\def\PYGZhy{\char`\-}
\def\PYGZsq{\char`\'}
\def\PYGZdq{\char`\"}
\def\PYGZti{\char`\~}
% for compatibility with earlier versions
\def\PYGZat{@}
\def\PYGZlb{[}
\def\PYGZrb{]}
\makeatother

\begin{document}

\maketitle
\tableofcontents
\phantomsection\label{index::doc}


Contents:
\phantomsection\label{index:module-pyontdd.lib.hadron}\index{pyontdd.lib.hadron (module)}\index{Hadron (class in pyontdd.lib.hadron)}

\begin{fulllineitems}
\phantomsection\label{index:pyontdd.lib.hadron.Hadron}\pysiglinewithargsret{\strong{class }\code{pyontdd.lib.hadron.}\bfcode{Hadron}}{\emph{data}, \emph{masses=None}, \emph{gamma\_type=None}, \emph{hadron\_type=None}, \emph{fit\_type=None}, \emph{lattice\_size=None}, \emph{config\_number=None}}{}
Class for Hadrons.
\begin{quote}\begin{description}
\item[{Parameters }] \leavevmode
\textbf{data} : array\_like
\begin{quote}

Numpy array of the data to be fit.
\end{quote}

\textbf{masses} : tuple
\begin{quote}

The bare masses of the valence quarks comprising the hadron e.g.:

\begin{Verbatim}[commandchars=\\\{\}]
\PYG{n}{masses}\PYG{o}{=}\PYG{p}{(}\PYG{l+m+mf}{0.005}\PYG{p}{,} \PYG{l+m+mf}{0.01}\PYG{p}{)}
\end{Verbatim}
\end{quote}

\textbf{gamma\_type} : string
\begin{quote}

The gamma structure of the propagators e.g.:

\begin{Verbatim}[commandchars=\\\{\}]
\PYG{n}{gamma\PYGZus{}type}\PYG{o}{=}\PYG{l+s}{\PYGZdq{}}\PYG{l+s}{AP}\PYG{l+s}{\PYGZdq{}}
\end{Verbatim}

for Axial-Pseudoscalar.
\end{quote}

\textbf{hadron\_type} : string
\begin{quote}

The type of Hadron that the data represents e.g.:

\begin{Verbatim}[commandchars=\\\{\}]
\PYG{n}{hadron\PYGZus{}type}\PYG{o}{=}\PYG{l+s}{\PYGZdq{}}\PYG{l+s}{PseudoscalarMeson}\PYG{l+s}{\PYGZdq{}}
\end{Verbatim}
\end{quote}

\textbf{fit\_type} : string
\begin{quote}

The way we fit the data:

\begin{Verbatim}[commandchars=\\\{\}]
\PYG{n}{fit\PYGZus{}type}\PYG{o}{=}\PYG{l+s}{\PYGZdq{}}\PYG{l+s}{Individual}\PYG{l+s}{\PYGZdq{}} \PYG{o+ow}{or} \PYG{l+s}{\PYGZdq{}}\PYG{l+s}{Simultaneous}\PYG{l+s}{\PYGZdq{}}
\end{Verbatim}
\end{quote}

\textbf{lattice\_size} : dict
\begin{quote}

Dict that specifies the size of the lattice e.g.:

\begin{Verbatim}[commandchars=\\\{\}]
\PYG{n}{lattice\PYGZus{}size} \PYG{o}{=} \PYG{p}{\PYGZob{}}\PYG{l+s}{\PYGZdq{}}\PYG{l+s}{x}\PYG{l+s}{\PYGZdq{}}\PYG{p}{:} \PYG{l+m+mi}{24}\PYG{p}{,} \PYG{l+s}{\PYGZdq{}}\PYG{l+s}{y}\PYG{l+s}{\PYGZdq{}}\PYG{p}{:} \PYG{l+m+mi}{24}\PYG{p}{,} \PYG{l+s}{\PYGZdq{}}\PYG{l+s}{z}\PYG{l+s}{\PYGZdq{}}\PYG{p}{:} \PYG{l+m+mi}{24}\PYG{p}{,} \PYG{l+s}{\PYGZdq{}}\PYG{l+s}{t}\PYG{l+s}{\PYGZdq{}}\PYG{p}{:} \PYG{l+m+mi}{64}\PYG{p}{,} \PYG{l+s}{\PYGZdq{}}\PYG{l+s}{s}\PYG{l+s}{\PYGZdq{}}\PYG{p}{:} \PYG{l+m+mi}{16}\PYG{p}{\PYGZcb{}}
\end{Verbatim}
\end{quote}

\textbf{config\_number} : int, optional
\begin{quote}

The configuration number that this propagator corresponds to.
\end{quote}

\end{description}\end{quote}
\paragraph{Methods}
\index{fit() (pyontdd.lib.hadron.Hadron method)}

\begin{fulllineitems}
\phantomsection\label{index:pyontdd.lib.hadron.Hadron.fit}\pysiglinewithargsret{\bfcode{fit}}{\emph{guess=None}, \emph{fit\_range=None}, \emph{covariant\_fit=False}, \emph{correlated\_fit=False}, \emph{inv\_covar=None}, \emph{error=None}}{}
Fit the Hadron based on the parameters given.

\end{fulllineitems}


\end{fulllineitems}

\index{HadronCharged (class in pyontdd.lib.hadron)}

\begin{fulllineitems}
\phantomsection\label{index:pyontdd.lib.hadron.HadronCharged}\pysiglinewithargsret{\strong{class }\code{pyontdd.lib.hadron.}\bfcode{HadronCharged}}{\emph{data}, \emph{masses=None}, \emph{charges=None}, \emph{gamma\_type=None}, \emph{hadron\_type=None}, \emph{fit\_type=None}, \emph{lattice\_size=None}, \emph{config\_number=None}}{}
Class for Charged Hadrons. Inherits all parameters from Hadron.
\begin{quote}\begin{description}
\item[{Parameters }] \leavevmode
\textbf{charges} : tuple
\begin{quote}

Tuple of charges of the quarks comprising the hadron in units of e/3 e.g. (2, -1)
\end{quote}

\end{description}\end{quote}


\strong{See also:}

\begin{description}
\item[{{\hyperref[index:pyontdd.lib.hadron.Hadron]{\code{Hadron}}}}] \leavevmode
HadronCharged inherits everything from this class.

\end{description}


\paragraph{Methods}
\index{get\_charges() (pyontdd.lib.hadron.HadronCharged method)}

\begin{fulllineitems}
\phantomsection\label{index:pyontdd.lib.hadron.HadronCharged.get_charges}\pysiglinewithargsret{\bfcode{get\_charges}}{}{}
Get the charges.
\begin{quote}\begin{description}
\item[{Returns }] \leavevmode
\textbf{tuple} :
\begin{quote}

Tuple of the charges.
\end{quote}

\end{description}\end{quote}

\end{fulllineitems}


\end{fulllineitems}

\phantomsection\label{index:module-pyontdd.lib.correlator}\index{pyontdd.lib.correlator (module)}\phantomsection\label{index:module-pyontdd.lib.io}\index{pyontdd.lib.io (module)}\phantomsection\label{index:module-pyontdd.lib.register}\index{pyontdd.lib.register (module)}\index{registerCorrelator() (in module pyontdd.lib.register)}

\begin{fulllineitems}
\phantomsection\label{index:pyontdd.lib.register.registerCorrelator}\pysiglinewithargsret{\code{pyontdd.lib.register.}\bfcode{registerCorrelator}}{\emph{f}}{}
Decorator to register a correlator type with the ones that CorrelatorFactory will search through.

\end{fulllineitems}

\phantomsection\label{index:module-pyontdd.lib.registered_types}\index{pyontdd.lib.registered\_types (module)}\index{RegisteredCorrelatorTypes (class in pyontdd.lib.registered\_types)}

\begin{fulllineitems}
\phantomsection\label{index:pyontdd.lib.registered_types.RegisteredCorrelatorTypes}\pysigline{\strong{class }\code{pyontdd.lib.registered\_types.}\bfcode{RegisteredCorrelatorTypes}}
A list of registered Correlator types that can be accessed by various functions/classes.
Initially empty, this is populated by pyontdd.lib.register.registerCorrelator(cls) as a decorator.

\end{fulllineitems}



\chapter{Indices and tables}
\label{index:indices-and-tables}\label{index:welcome-to-pyontdd-s-documentation}\begin{itemize}
\item {} 
\emph{genindex}

\item {} 
\emph{modindex}

\item {} 
\emph{search}

\end{itemize}


\renewcommand{\indexname}{Python Module Index}
\begin{theindex}
\def\bigletter#1{{\Large\sffamily#1}\nopagebreak\vspace{1mm}}
\bigletter{p}
\item {\texttt{pyontdd.lib.correlator}}, \pageref{index:module-pyontdd.lib.correlator}
\item {\texttt{pyontdd.lib.hadron}}, \pageref{index:module-pyontdd.lib.hadron}
\item {\texttt{pyontdd.lib.io}}, \pageref{index:module-pyontdd.lib.io}
\item {\texttt{pyontdd.lib.register}}, \pageref{index:module-pyontdd.lib.register}
\item {\texttt{pyontdd.lib.registered\_types}}, \pageref{index:module-pyontdd.lib.registered_types}
\end{theindex}

\renewcommand{\indexname}{Index}
\printindex
\end{document}
